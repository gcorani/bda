% Options for packages loaded elsewhere
\PassOptionsToPackage{unicode}{hyperref}
\PassOptionsToPackage{hyphens}{url}
%
\documentclass[
]{article}
\usepackage{amsmath,amssymb}
\usepackage{lmodern}
\usepackage{iftex}
\ifPDFTeX
  \usepackage[T1]{fontenc}
  \usepackage[utf8]{inputenc}
  \usepackage{textcomp} % provide euro and other symbols
\else % if luatex or xetex
  \usepackage{unicode-math}
  \defaultfontfeatures{Scale=MatchLowercase}
  \defaultfontfeatures[\rmfamily]{Ligatures=TeX,Scale=1}
\fi
% Use upquote if available, for straight quotes in verbatim environments
\IfFileExists{upquote.sty}{\usepackage{upquote}}{}
\IfFileExists{microtype.sty}{% use microtype if available
  \usepackage[]{microtype}
  \UseMicrotypeSet[protrusion]{basicmath} % disable protrusion for tt fonts
}{}
\makeatletter
\@ifundefined{KOMAClassName}{% if non-KOMA class
  \IfFileExists{parskip.sty}{%
    \usepackage{parskip}
  }{% else
    \setlength{\parindent}{0pt}
    \setlength{\parskip}{6pt plus 2pt minus 1pt}}
}{% if KOMA class
  \KOMAoptions{parskip=half}}
\makeatother
\usepackage{xcolor}
\usepackage[margin=1in]{geometry}
\usepackage{longtable,booktabs,array}
\usepackage{calc} % for calculating minipage widths
% Correct order of tables after \paragraph or \subparagraph
\usepackage{etoolbox}
\makeatletter
\patchcmd\longtable{\par}{\if@noskipsec\mbox{}\fi\par}{}{}
\makeatother
% Allow footnotes in longtable head/foot
\IfFileExists{footnotehyper.sty}{\usepackage{footnotehyper}}{\usepackage{footnote}}
\makesavenoteenv{longtable}
\usepackage{graphicx}
\makeatletter
\def\maxwidth{\ifdim\Gin@nat@width>\linewidth\linewidth\else\Gin@nat@width\fi}
\def\maxheight{\ifdim\Gin@nat@height>\textheight\textheight\else\Gin@nat@height\fi}
\makeatother
% Scale images if necessary, so that they will not overflow the page
% margins by default, and it is still possible to overwrite the defaults
% using explicit options in \includegraphics[width, height, ...]{}
\setkeys{Gin}{width=\maxwidth,height=\maxheight,keepaspectratio}
% Set default figure placement to htbp
\makeatletter
\def\fps@figure{htbp}
\makeatother
\setlength{\emergencystretch}{3em} % prevent overfull lines
\providecommand{\tightlist}{%
  \setlength{\itemsep}{0pt}\setlength{\parskip}{0pt}}
\setcounter{secnumdepth}{-\maxdimen} % remove section numbering
\ifLuaTeX
  \usepackage{selnolig}  % disable illegal ligatures
\fi
\IfFileExists{bookmark.sty}{\usepackage{bookmark}}{\usepackage{hyperref}}
\IfFileExists{xurl.sty}{\usepackage{xurl}}{} % add URL line breaks if available
\urlstyle{same} % disable monospaced font for URLs
\hypersetup{
  hidelinks,
  pdfcreator={LaTeX via pandoc}}

\author{}
\date{\vspace{-2.5em}}

\begin{document}

\hypertarget{section}{%
\section{}\label{section}}

\hypertarget{confronto-fra-due-proporzioni}{%
\subsection{Confronto fra due
proporzioni}\label{confronto-fra-due-proporzioni}}

\hypertarget{confronto-fra-due-proporzioni-1}{%
\section{Confronto fra due
proporzioni}\label{confronto-fra-due-proporzioni-1}}

\begin{itemize}
\tightlist
\item
  Vogliamo confrontare i parametri \(\pi_1\) e \(\pi_2\) che
  caratterizzano due distribuzioni binomiali.
\end{itemize}

\bigskip

\begin{itemize}
\tightlist
\item
  Il test si basa su una approssimazione per campioni ampi per i quali
  \(p_1 = \frac{X_1}{n_1}\) e \(p_1 = \frac{X_2}{n_2}\) sono normalmente
  distribuite.
\end{itemize}

\bigskip

\begin{itemize}
\tightlist
\item
  Entrambi i campioni devono contenere almeno 5 successi e 5 insuccessi.
\end{itemize}

\bigskip

\begin{itemize}
\tightlist
\item
  Questo permette di applicare l'approssimazione normale alla binomiale
  per entrambe le popolazioni.
\end{itemize}

\hypertarget{confronto-fra-due-proporzioni-2}{%
\section{Confronto fra due
proporzioni}\label{confronto-fra-due-proporzioni-2}}

\begin{itemize}
\tightlist
\item
  Abbiamo estratto due campioni di dimensione \(n_1\) e \(n_2\) dalle
  due popolazioni, il cui numero di successi è rispettivamente \(X_1\) e
  \(X_2\).
\end{itemize}

\bigskip

\begin{itemize}
\tightlist
\item
  Il test di uguaglianza a due code è:
\end{itemize}

\begin{align*}
H_0 \; & : \pi_1 = \pi_2 \\
H_1 \; & : \pi_1 \neq \pi_2
\end{align*}

\hypertarget{statistica-del-test}{%
\section{Statistica del test}\label{statistica-del-test}}

\begin{align*}
Z 
= \frac{(p_1 - p_2)}{\sqrt{\bar{p}(1-\bar{p}) \left( \frac{1}{n_1} + \frac{1}{n_2} \right)}} \\
\end{align*}

che sotto \(H_0\) si distribuisce come una \(N(0,1)\).

\bigskip

\begin{itemize}
\tightlist
\item
  Nel caso di test a una coda, la statistica rimane identica ma cambia
  la regione di rifiuto.
\end{itemize}

\hypertarget{riassunto-regioni-di-rifiuto}{%
\section{Riassunto regioni di
rifiuto}\label{riassunto-regioni-di-rifiuto}}

\begin{longtable}[]{@{}
  >{\centering\arraybackslash}p{(\columnwidth - 6\tabcolsep) * \real{0.2872}}
  >{\centering\arraybackslash}p{(\columnwidth - 6\tabcolsep) * \real{0.4574}}
  >{\centering\arraybackslash}p{(\columnwidth - 6\tabcolsep) * \real{0.2234}}
  >{\raggedright\arraybackslash}p{(\columnwidth - 6\tabcolsep) * \real{0.0319}}@{}}
\toprule()
\begin{minipage}[b]{\linewidth}\centering
\(H_1\)
\end{minipage} & \begin{minipage}[b]{\linewidth}\centering
Regione di rifiuto
\end{minipage} & \begin{minipage}[b]{\linewidth}\centering
p-value
\end{minipage} & \begin{minipage}[b]{\linewidth}\raggedright
\end{minipage} \\
\midrule()
\endhead
\(\pi_1 \neq \pi_2\) & \(z < z_{\alpha/2}\) e \(z > z_{1-\alpha/2}\) &
\(2 (1-\Phi(|z|))\) & \\
\(\pi_1 > \pi_2\) & \(z > z_{1-\alpha}\) & \(1-\Phi(z)\) & \\
\(\pi_1 < \pi_2\) & \(z < z_{\alpha}\) & \(\Phi(z)\) & \\
\bottomrule()
\end{longtable}

\hypertarget{intervallo-di-confidenza-di-pi_1---pi_2}{%
\section{\texorpdfstring{Intervallo di confidenza di
\(\pi_1 - \pi_2\)}{Intervallo di confidenza di \textbackslash pi\_1 - \textbackslash pi\_2}}\label{intervallo-di-confidenza-di-pi_1---pi_2}}

I valori plausibili di \(\pi_1 - \pi_2\) sono contenuti nell'intervallo:
\[
p_1 - p_2 \pm z_{1-\alpha/2} \sqrt{ \frac{p_1 (1-p_1)}{n_1} + \frac{p_2 (1-p_2)}{n_2}}
\]

\bigskip

\begin{itemize}
\tightlist
\item
  In questo caso non c'è una perfetta corrispondenza tra CI e test a due
  code, perchè il test ed il CI stimano l'errore standard di
  \(\pi_1 - pi_2\) in modo (leggermente) diverso.
\end{itemize}

\hypertarget{esempio-valutare-lefficacia-di-un-farmaco}{%
\section{Esempio: valutare l'efficacia di un
farmaco}\label{esempio-valutare-lefficacia-di-un-farmaco}}

\begin{itemize}
\tightlist
\item
  Per valutare l'efficacia di un farmaco si svolge un \emph{randomized
  trial}.
\end{itemize}

\bigskip

\begin{itemize}
\tightlist
\item
  In modo casuale ad alcuni pazienti viene somministrato il farmaco; ad
  altri il placebo.
\end{itemize}

\bigskip

\begin{itemize}
\tightlist
\item
  Alla fine del periodo di cura, è necessario analizzare se c'è una
  differenza statisticamente significativa fra i due gruppi.
\end{itemize}

\hypertarget{esempio-valutare-lefficacia-di-un-farmaco-1}{%
\section{Esempio: valutare l'efficacia di un
farmaco}\label{esempio-valutare-lefficacia-di-un-farmaco-1}}

\begin{itemize}
\tightlist
\item
  Il gruppo placebo contiene 227 pazienti, di cui 163 guariti al termine
  del periodo.
\end{itemize}

\bigskip

\begin{itemize}
\tightlist
\item
  Il gruppo farmaco contiene 262 pazienti, di cui 154 guariti al termine
  del periodo.
\end{itemize}

\bigskip

\begin{itemize}
\tightlist
\item
  Il farmaco è significativamente più efficace del placebo?
\end{itemize}

\hypertarget{valutare-lefficacia-di-un-farmaco}{%
\section{Valutare l'efficacia di un
farmaco}\label{valutare-lefficacia-di-un-farmaco}}

\begin{itemize}
\tightlist
\item
  Entrambi i gruppi contengono almeno 5 successi e 5 insuccessi;
  possiamo quindi fare il test che usa l'approssimazione per campioni
  larghi.
\end{itemize}

\bigskip

\begin{itemize}
\tightlist
\item
  Vogliamo provare a dimostrare l'efficacia del farmaco, quindi facciamo
  il test:
\end{itemize}

\begin{align*}
H_0 \; : \pi_{\text{farmaco}} \leq \pi_{\text{placebo}}\\
H_1 \; : \pi_{\text{farmaco}} > \pi_{\text{placebo}}\\
\end{align*}

\begin{itemize}
\tightlist
\item
  Vogliamo forte evidenza che il farmaco sia efficace, quindi usiamo
  \(\alpha=0.01\).
\end{itemize}

\hypertarget{valutare-lefficacia-di-un-farmaco-1}{%
\section{Valutare l'efficacia di un
farmaco}\label{valutare-lefficacia-di-un-farmaco-1}}

\begin{itemize}
\tightlist
\item
  La regione di rifiuto del test contiene valori \emph{positivi} di
  \(p_{\text{farmaco}} - p_{\text{placebo}}\) e quindi della statistica.
\end{itemize}

\bigskip

\begin{itemize}
\tightlist
\item
  Regione di rifiuto: \(Z_0 > \Phi^{-1}{.99}\) = 1.64
\end{itemize}

\hypertarget{valutare-lefficacia-di-un-farmaco-2}{%
\section{Valutare l'efficacia di un
farmaco}\label{valutare-lefficacia-di-un-farmaco-2}}

\bigskip

\begin{align*}
p_{\text{farmaco}} & = 163/227 = 0.72 \\
p_{\text{placebo}} & = 154/262 = 0.59 \\
\bar{p} & = (163 + 154)/(227 + 262) = 0.65\\
Z & = (p_{\text{farmaco}} - p_{\text{placebo}}) / \sqrt{ \bar{p} \cdot (1-\bar{p}) \cdot 1/n} = 3.01\\
\text{p-value} & = $1-\Phi(3.01)$ = 0.0013
\end{align*}

\begin{itemize}
\tightlist
\item
  Il p-value è un ordine di grandezza più piccolo di \(\alpha\)
\end{itemize}

\bigskip

\begin{itemize}
\tightlist
\item
  Il CI è {[}-0.3, 0.1{]} che non contiene lo 0. RIVEDERE.
\end{itemize}

\hypertarget{esercizio}{%
\section{Esercizio}\label{esercizio}}

\begin{itemize}
\tightlist
\item
  Un processo produce cuscinetti per l'albero motore.
\end{itemize}

\bigskip

\begin{itemize}
\tightlist
\item
  Si preleva un campione di 85 cuscinetti, che risulta contenere 12
  non-conformi.
\end{itemize}

\bigskip

\begin{itemize}
\tightlist
\item
  Il processo produttivo viene quindi rivisto. Si preleva un nuovo
  campione di 85 cuscinetti, che risulta contenere 8 non-conformi.
\end{itemize}

\bigskip

\begin{itemize}
\tightlist
\item
  Possiamo concludere con confidenza del 95\% che la frazione di
  non-conformi è significativamente decresciuta?
\end{itemize}

\end{document}
