% Options for packages loaded elsewhere
\PassOptionsToPackage{unicode}{hyperref}
\PassOptionsToPackage{hyphens}{url}
%
\documentclass[
  13pt,
  ignorenonframetext,
]{beamer}
\usepackage{pgfpages}
\setbeamertemplate{caption}[numbered]
\setbeamertemplate{caption label separator}{: }
\setbeamercolor{caption name}{fg=normal text.fg}
\beamertemplatenavigationsymbolsempty
% Prevent slide breaks in the middle of a paragraph
\widowpenalties 1 10000
\raggedbottom
\setbeamertemplate{part page}{
  \centering
  \begin{beamercolorbox}[sep=16pt,center]{part title}
    \usebeamerfont{part title}\insertpart\par
  \end{beamercolorbox}
}
\setbeamertemplate{section page}{
  \centering
  \begin{beamercolorbox}[sep=12pt,center]{part title}
    \usebeamerfont{section title}\insertsection\par
  \end{beamercolorbox}
}
\setbeamertemplate{subsection page}{
  \centering
  \begin{beamercolorbox}[sep=8pt,center]{part title}
    \usebeamerfont{subsection title}\insertsubsection\par
  \end{beamercolorbox}
}
\AtBeginPart{
  \frame{\partpage}
}
\AtBeginSection{
  \ifbibliography
  \else
    \frame{\sectionpage}
  \fi
}
\AtBeginSubsection{
  \frame{\subsectionpage}
}
\usepackage{amsmath,amssymb}
\usepackage{lmodern}
\usepackage{iftex}
\ifPDFTeX
  \usepackage[T1]{fontenc}
  \usepackage[utf8]{inputenc}
  \usepackage{textcomp} % provide euro and other symbols
\else % if luatex or xetex
  \usepackage{unicode-math}
  \defaultfontfeatures{Scale=MatchLowercase}
  \defaultfontfeatures[\rmfamily]{Ligatures=TeX,Scale=1}
\fi
\usetheme[]{metropolis}
% Use upquote if available, for straight quotes in verbatim environments
\IfFileExists{upquote.sty}{\usepackage{upquote}}{}
\IfFileExists{microtype.sty}{% use microtype if available
  \usepackage[]{microtype}
  \UseMicrotypeSet[protrusion]{basicmath} % disable protrusion for tt fonts
}{}
\makeatletter
\@ifundefined{KOMAClassName}{% if non-KOMA class
  \IfFileExists{parskip.sty}{%
    \usepackage{parskip}
  }{% else
    \setlength{\parindent}{0pt}
    \setlength{\parskip}{6pt plus 2pt minus 1pt}}
}{% if KOMA class
  \KOMAoptions{parskip=half}}
\makeatother
\usepackage{xcolor}
\newif\ifbibliography
\setlength{\emergencystretch}{3em} % prevent overfull lines
\providecommand{\tightlist}{%
  \setlength{\itemsep}{0pt}\setlength{\parskip}{0pt}}
\setcounter{secnumdepth}{-\maxdimen} % remove section numbering
\usepackage{MonashWhite}
\usepackage{amsmath,bm,booktabs,tikz,xcolor}
\usepackage{animate}

\setbeamercolor{description item}{fg=Orange}

\def\pred#1#2#3{\hat{#1}_{#2|#3}}
\def\damped{$_\text{d}$}
\def\h+{h_{m}^{+}}
\def\st#1{\rlap{#1}\textcolor{red}{\rule{1cm}{0.1cm}}}

\graphicspath{{figs/}}

% Monash title page
\setbeamerfont{title}{series=\bfseries,parent=structure,size={\fontsize{26}{30}}}
\setbeamertemplate{title page}
{
%\placefig{-0.01}{-0.01}{width=1.01\paperwidth,height=1.01\paperheight}{figs/MonashTitleSlide}
\begin{textblock}{7.5}(1,2)\fontsize{20}{30}\sf
{\raggedright\usebeamerfont{title}\par\inserttitle}
\end{textblock}
\begin{textblock}{7.5}(1,7.3)
{\raggedright{\insertauthor}\\[0.2cm]
\insertdate}
\end{textblock}}

\setlength\abovedisplayskip{0pt}

\AtBeginSection[] {
  \begin{frame}
  \frametitle{Outline}
    \tableofcontents[currentsection]
  \end{frame}
}

\usetikzlibrary{shapes,arrows}
\tikzstyle{decision} = [diamond, draw, fill=blue!20,
    text width=4.5em, text badly centered, node distance=4cm, inner sep=0pt]
\tikzstyle{block} = [rectangle, draw, fill=blue!20,
    text width=5cm, text centered, rounded corners, minimum height=4em]
\tikzstyle{line} = [draw, thick, -latex']
%\tikzstyle{line} = [->,thi
\tikzstyle{cloud} = [draw, ellipse,fill=red!20, node distance=3cm,
    minimum height=2em, text centered]
\tikzstyle{connector} = [->,thick]

\def\E{\text{E}}
\def\V{\text{Var}}
\def\up#1{\raisebox{-0.3cm}{#1}}

\setlength{\emergencystretch}{0em}
\setlength{\parskip}{0pt}
\def\fullwidth#1{\vspace*{-0.1cm}\par\centerline{\includegraphics[width=12.8cm]{#1}}}
\def\fullheight#1{\vspace*{-0.1cm}\par\centerline{\includegraphics[height=8.5cm]{#1}}}

\fontsize{13}{15}\sf
\usepackage[scale=0.85]{sourcecodepro}
\DisableLigatures{encoding = T1, family = tt*}

\setbeamertemplate{navigation symbols}{}
\setbeamertemplate{footline}[page number]
\ifLuaTeX
  \usepackage{selnolig}  % disable illegal ligatures
\fi
\IfFileExists{bookmark.sty}{\usepackage{bookmark}}{\usepackage{hyperref}}
\IfFileExists{xurl.sty}{\usepackage{xurl}}{} % add URL line breaks if available
\urlstyle{same} % disable monospaced font for URLs
\hypersetup{
  pdftitle={Bayesian Data Analysis course - Project work},
  pdfauthor={Giorgio Corani - SUPSI},
  hidelinks,
  pdfcreator={LaTeX via pandoc}}

\title{Bayesian Data Analysis course - Project work}
\author{Giorgio Corani - SUPSI}
\date{Bayesian Data analysis and Probabilistic Programming}

\begin{document}
\frame{\titlepage}

\begin{frame}
\begin{block}{Project work details}
\protect\hypertarget{project-work-details}{}
The project requires choosing a data set and analyzing it using the
methodologies learned during the course.
\end{block}
\end{frame}

\begin{frame}{Groups}
\protect\hypertarget{groups}{}
\begin{itemize}
\tightlist
\item
  The project is done by a group of two persons.
\item
  In case of strong disagreement with your mate, you can ask to perform
  the project on your own (the amount of the work is however the same of
  the 2-persons project).
\end{itemize}
\end{frame}

\begin{frame}{Evaluation}
\protect\hypertarget{evaluation}{}
\begin{itemize}
\item
  The final grade of the course is:

  \begin{itemize}
  \tightlist
  \item
    75\%: evaluation of the project
  \item
    25\% evaluation of the assignment
  \end{itemize}
\item
  \textbf{fare presentazione o solo orale?? }
\item
  Suspected plagiarism will be investigated and can lead to a null
  grade.
\end{itemize}
\end{frame}

\begin{frame}[fragile]{Deadlines}
\protect\hypertarget{deadlines}{}
\begin{itemize}
\item
  Group registration by \textbf{date}, using the moodle choosing tool.
\item
  Report submission (no more than 15 pages) by \textbf{date} using
  moodle.
\item
  Oral presentation of the project: \textbf{date}.
\item
  After the presentation we will make you some question about your work.
\item
  TA session queue is also for project questions.
\end{itemize}

\begin{block}{TA sessions}
\protect\hypertarget{ta-sessions}{}
The groups will get help for the project work in
\href{assignments.html\#TA_sessions}{TA sessions}. When there are no
weekly assignments, the TA sessions are still organized for helping in
the project work.
\end{block}

\begin{block}{Evaluation}
\protect\hypertarget{evaluation-1}{}
The project work's evaluation consists of

\begin{itemize}
\tightlist
\item
  peergraded project report (40\%) (within peergrade submission 80\% and
  feedack 20\%)
\item
  presentation and oral exam graded by the course staff (60\%)

  \begin{itemize}
  \tightlist
  \item
    clarity of slides + use of figures
  \item
    clarity of oral presentation + flow of the presentation
  \item
    all required parts included (not necessarily all in main slides, but
    it needs to be clear that all required steps were performed)
  \item
    accuracy of use of terms (oral exam)
  \item
    responses to questions (oral exam)
  \end{itemize}
\end{itemize}
\end{block}

\begin{block}{Project report}
\protect\hypertarget{project-report}{}
In the project report you practice presenting the problem and data
analysis results, which means that minimal listing of code and figures
is not a good report. There are different levels for how data analysis
project could be reported. This report should be more than a summary of
results without workflow steps. While describing the steps and decisions
made during the workflow, to keep the report readable some of the
diagnostic outputs and code can be put in the appendix. If you are
uncertain you can ask TAs in TA sessions whether you are on a good level
of amount of details.

The report should include

\begin{enumerate}
\tightlist
\item
  Introduction describing

  \begin{itemize}
  \tightlist
  \item
    the motivation
  \item
    the problem
  \item
    and the main modeling idea.
  \item
    Showing some illustrative figure is recommended.
  \end{itemize}
\item
  Description of the data and the analysis problem. Provide information
  where the data was obtained, and if it has been previously used in
  some online case study and how your analysis differs from the existing
  analyses.
\item
  Description of at least two models, for example:

  \begin{itemize}
  \tightlist
  \item
    non hierarchical and hierarchical,
  \item
    linear and non linear,
  \item
    variable selection with many models.
  \end{itemize}
\item
  Informative or weakly informative priors, and justification of their
  choices.
\item
  Stan, rstanarm or brms code.
\item
  How to the Stan model was run, that is, what options were used. This
  is also more clear as combination of textual explanation and the
  actual code line.
\item
  Convergence diagnostics (\(\widehat{R}\), ESS, divergences) and what
  was done if the convergence was not good with the first try.
\item
  Posterior predictive checks and what was done to improve the model.
\item
  Model comparison (e.g.~with LOO-CV).
\item
  Predictive performance assessment if applicable (e.g.~classification
  accuracy) and evaluation of practical usefulness of the accuracy.
\item
  Sensitivity analysis with respect to prior choices (i.e.~checking
  whether the result changes a lot if prior is changed)
\item
  Discussion of issues and potential improvements.
\item
  Conclusion what was learned from the data analysis.
\item
  Self-reflection of what the group learned while making the project.
\end{enumerate}
\end{block}

\begin{block}{Project presentation}
\protect\hypertarget{project-presentation}{}
In addition to the submitted report, each project must be presented by
the authoring group, according to the following guidelines:

\begin{itemize}
\tightlist
\item
  The presentation should be high level but sufficiently detailed
  information should be readily available to help answering questions
  from the audience.
\item
  The duration of the presentation should be 10 minutes (groups of 1-2
  students) or 15 minutes (groups of 3 students).
\item
  At the end of the presentation there will be an extra 5-10 minutes of
  questions by anyone in the audience or two members of the course staff
  who are presenr. The questions from lecturer/TAs can be considered as
  an oral exam questions, and if answers to these questions reveal weak
  knowledge of the methods and workflow steps which should be part of
  the project, that can reduce the grade.
\item
  Grading will be done by the two members of the course staff using
  standardized grading instructions.
\end{itemize}

Specific recommendations for the presentations include:

\begin{itemize}
\tightlist
\item
  The first slide should include project's title and group members'
  names.
\item
  The chosen statistical model(s), including observation model and
  priors, must be explained and justified,
\item
  Make sure the font is big enough to be easily readable by the
  audience. This includes figure captions, legends and axis information,
\item
  The last slide should be a summary or take-home-messages and include
  contact information or link to a further information. (The grade will
  be reduced by one if the last slide has only something like ``Thank
  you'' or ``Questions?''),
\item
  In general, the best presentations are often given by teams that have
  frequently attended TA sessions and gotten feedback, so we strongly
  recommend attending these sessions.
\end{itemize}

More details on the presentation sessions

\begin{itemize}
\tightlist
\item
  If you don't have microphone or video camera (e.g.~in your laptop or
  mobile phone) then we'll arrange your presentation on campus in period
  III.
\item
  If you reserved a presentation slot but need to cancel, do it asap.
\item
  Zoom meeting link for all time slots available in the course chat.
\item
  As we have many presentation in each slot join the meeting in time.
  Late arrivals will lower the grade. Very late arrivals will fail the
  presentation and can present in period III.
\item
  Presenting group needs to have video and audio on.
\item
  It is easiest if just one from the group shares the slides, but it is
  expected that all group members present some part of the presentation
  orally.
\item
  Presentation time is 10 min for 1-2 person groups and 15min for 3
  person groups
\item
  Time limit is strict. It's good idea to practice the talk so that you
  get the timing right. Staff will announce 2min and 1min left and time
  ended. Going overtime reduces the grade.
\item
  After the presentation there will be 5min for questions, answers, and
  feedback.
\item
  Each student has to come up with at least one question during the
  session. Students can ask more questions. Questions by students are
  posted in chat, and they can be posted already during the
  presentation.
\item
  Staff Will ask further questions (kind of oral exam)
\item
  Grading of the project presentation takes int account

  \begin{itemize}
  \tightlist
  \item
    clarity of slides + use of figures
  \item
    clarity of oral presentation + flow of the presentation
  \item
    all required parts included (not necessarily all in main slides, but
    it needs to be clear that all required steps were performed)
  \item
    accuracy of use of terms (oral exam)
  \item
    responses to questions (oral exam)
  \end{itemize}
\item
  Students will also self-evaluate their project. After the presentation
  each student who just presented sends a private message to one of the
  staff members with a self evaluation grade from themselves and for
  each group member (if applicable).
\end{itemize}
\end{block}

\begin{block}{Data sets}
\protect\hypertarget{data-sets}{}
As some data sets have been overused for these particular goals, note
that the following ones are forbidden in this work (more can be added to
this list so make sure to check it regularly):

\begin{itemize}
\tightlist
\item
  extremly common data sets like titanic, mtcars, iris
\item
  Baseball batting (used by Bob Carpenter's StanCon case study).
\item
  Data sets used in the course demos
\end{itemize}

It's best to use a dataset for which there is no ready made analysis in
internet, but if you choose a dataset used already in some online case
study, provide the link to previous studies and report how your analysis
differs from those (for example if someone has made non-Bayesian
analysis and you do the full Bayesian analysis).

Depending on the model and the structure of the data, a good data set
would have more than 100 observations but less than 1 million. If you
know an interesting big data set, you can use a smaller subset of the
data to keep the computation times feasible. It would be good that the
data has some structure, so that it is sensible to use
multilevel/hierarchical models.
\end{block}

\begin{block}{Model requirements}
\protect\hypertarget{model-requirements}{}
\begin{itemize}
\tightlist
\item
  Every parameter needs to have an explicit proper prior. Improper flat
  priors are not allowed.
\item
  A hierarchical model is a model where the prior of certain parameter
  contain other parameters that are also estimated in the model. For
  instance, \texttt{b\ \textasciitilde{}\ normal(mu,\ sigma)},
  \texttt{mu\ \textasciitilde{}\ normal(0,\ 1)},
  \texttt{sigma\ \textasciitilde{}\ exponential(1)}.
\item
  Do not impose hard constrains on a parameter unless they are natural
  to them. \texttt{uniform(a,\ b)} should not be used unless the
  boundaries are really logical boundaries and values beyond the
  boundaries are completely impossible.
\item
  At least some models should include covariates. Modelling the outcome
  without predictors is likely too simple for the project.
\item
  \texttt{brms} can be used, but the Stan code must be included, briefly
  commented, and all priors need to be checked from the Stan code and
  adjusted to be weakly informative based on some justified explanation.
\end{itemize}
\end{block}

\begin{block}{Some examples}
\protect\hypertarget{some-examples}{}
The following case study examples demonstrate how text, equations,
figures, and code, and inference results can be included in one report.
These examples don't necessarily have all the workflow steps required in
your report, but different steps are illustrated in different case
studies and you can get good ideas for your report just by browsing
through them.

\begin{itemize}
\tightlist
\item
  \href{demos.html}{BDA R and Python demos} are quite minimal in
  description of the data and discussion of the results, but show many
  diagnostics and basic plots.
\item
  Some \href{https://mc-stan.org/users/documentation/case-studies}{Stan
  case studies} focus on some specific methods, but there are many case
  studies that are excellent examples for this course. They don't
  include all the steps required in this course, but are good examples
  of writing. Some of them are longer or use more advanced models than
  required in this course.

  \begin{itemize}
  \tightlist
  \item
    \href{https://mc-stan.org/users/documentation/case-studies/boarding_school_case_study.html}{Bayesian
    workflow for disease transmission modeling in Stan}
  \item
    \href{https://mc-stan.org/users/documentation/case-studies/model-based_causal_inference_for_RCT.html}{Model-based
    Inference for Causal Effects in Completely Randomized Experments}
  \item
    \href{https://mc-stan.org/users/documentation/case-studies/bball-hmm.html}{Tagging
    Basketball Events with HMM in Stan}
  \item
    \href{https://mc-stan.org/users/documentation/case-studies/golf.html}{Model
    building and expansion for golf putting}
  \item
    \href{https://mc-stan.org/users/documentation/case-studies/dyadic_irt_model.html}{A
    Dyadic Item Response Theory Model}
  \item
    \href{https://mc-stan.org/users/documentation/case-studies/lotka-volterra-predator-prey.html}{Predator-Prey
    Population Dynamics: the Lotka-Volterra model in Stan}
  \end{itemize}
\item
  Some \href{https://github.com/stan-dev/stancon_talks}{StanCon case
  studies} (scroll down) can also provide good project ideas.
\end{itemize}
\end{block}
\end{frame}

\end{document}
